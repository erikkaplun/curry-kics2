\documentclass[english]{lni}

%\IfFileExists{latin1.sty}{\usepackage{latin1}}{\usepackage{isolatin1}}

\usepackage{graphicx}

\author{
Michael Hanus
\quad
Bj{\"o}rn Peem{\"o}ller
\quad
Fabian Reck \\
\\
Institut f\"ur Informatik, CAU Kiel, D-24098 Kiel, Germany \\
\texttt{\{mh|bjp|fre\}@informatik.uni-kiel.de}
}
\title{Search Strategies for Functional Logic Programming}
\begin{document}
\maketitle

\begin{abstract}
In this paper we discuss our practical experiences
with the use of different search strategies
in functional logic programs.
In particular, we show that complete strategies,
like breadth-first search or iterative deepening search,
are a reasonable alternative to incomplete strategies, like depth-first
search, that have been favored in the past for logic programming.
\end{abstract}

\section{Introduction}

Functional logic languages combine the most important
features of functional and logic programming in a single language
(see \cite{AntoyHanus10CACM,Hanus07ICLP} for recent surveys).
In particular, they provide higher-order functions and demand-driven
evaluation from functional programming together with logic programming features
like non-deterministic search and computing with partial information
(logic variables).
This combination
led to new design patterns \cite{AntoyHanus02FLOPS,AntoyHanus11WFLP}
and better abstractions for application programming,
e.g., as shown for programming with databases
\cite{BrasselHanusMueller08PADL,Fischer05},
GUI programming \cite{Hanus00PADL},
web programming \cite{Hanus01PADL,Hanus06PPDP,HanusKoschnicke10PADL},
or string parsing \cite{CaballeroLopez99}.
Moreover, it is also a good basis
to teach the ideas of functional and logic programming,
or declarative programming in general,
with a single computation model and programming language
\cite{Hanus97DPLE}.

An important feature of logic programming languages
is non-deterministic search.
In Prolog, which is still the standard language for logic programming,
non-deterministic search is implemented via backtracking,
which corresponds to a depth-first search traversal
through the SLD proof tree \cite{Lloyd87}.


classical view: LP = lvars, unif + Backtracking

-> FP view: add backtracking to obtain LP

beginners: backtracking considered harmful

better: complete strategies

here: complete strategies reasonable, thus, should be the default

\section{Curry}

A short standard intro to Curry.

Mention that search strategy is not fixed
 (discuss top-level vs. encaps. search)

\section{Search Strategies}

Show enacpsulated search, i.e., introduce search tree
and implementation of search strategies in search trees
(maybe BFS and IDS in the appendix).

\section{Benchmarks}

Compare DFS, BFS, IDS (top-level and encapsulated, all solutions and first sol)

Hopefully, they show what we want...

\section{Conclusions}

If efficiency important and DFS safe (e.g., finite search space): use DFS

Future work: approximate safeness of DFS


\bibliography{mh}
%\bibliography{paper}

\end{document}
