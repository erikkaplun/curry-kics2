\section{ERD2Curry: A Tool to Generate Programs from ER Specifications}
\label{sec-erd2curry}

ERD2Curry\index{ERD2Curry}\index{database programming}
is a tool to generate Curry code to access and manipulate data
persistently stored from
entity relationship diagrams.\index{entity relationship diagrams}
The idea of this tool is described in detail in
\cite{BrasselHanusMueller08PADL}.
Thus, we describe only the basic steps to use this tool
in the following.

If one creates an entity relationship diagram (ERD)
with the Umbrello UML Modeller, one has to store its
XML description in XMI format (as offered by Umbrello)
in a file, e.g., \ccode{myerd.xmi}.
This description can be compiled into a Curry program by the
command\pindex{erd2curry}
\begin{curry}
erd2curry myerd.xmi
\end{curry}
(\code{erd2curry} is a program stored in \code{\cyshome/bin}
where \cyshome is the installation directory of \CYS;
see Section~\ref{sec-install}).
If \code{MyData} is the name of the ERD, the Curry program file
\ccode{MyData.curry} is generated containing all the necessary
database access code as described in \cite{BrasselHanusMueller08PADL}.

If one does not want to use the Umbrello UML Modeller,
one can also create a textual description of the ERD
as a Curry term of type \code{ERD}
(w.r.t.\ the type definition given in module
\code{\cyshome/tools/erd2curry/ERD.curry})
and store it in some file, e.g., \ccode{myerd.term}.
This description can be compiled into a Curry program by the
command\pindex{erd2curry}
\begin{curry}
erd2curry -t myerd.term
\end{curry}
%
There is also the possibility to visualize an ERD term
as a graph with the graph visualization program \code{dotty}
(for this purpose, it might be necessary to adapt the definition
of \code{dotviewcommand} in your \ccode{.kics2rc} file,
see Section~\ref{sec-customization},
according to your local environment).
This can be done by the command
\begin{curry}
erd2curry -v myerd.term
\end{curry}

\paragraph{Inclusion in the Curry application:}
To compile the generated database code, either
include the directory \code{\cyshome/tools/erd2curry}
into your Curry load path
(e.g., by setting  the environment variable
\ccode{CURRYPATH}\pindex{CURRYPATH}, see also Section~\ref{sec-modules})
or copy the file
\code{\cyshome/tools/erd2curry/ERDGeneric.curry}
into the directory of the generated database code.
