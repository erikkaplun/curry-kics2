\addcontentsline{toc}{section}{Preface}
\section*{Preface}

This document describes \CYS (\textbf{Ki}el \textbf{C}urry \textbf{S}ystem
Version \textbf{2}),
an implementation of the multi-paradigm language Curry
\cite{Hanus97POPL,Hanus06Curry} that is
based on compiling Curry programs into Haskell programs.
Curry is a universal programming language aiming at the amalgamation
of the most important declarative programming paradigms,
namely functional programming and logic programming.
Curry combines in a seamless way features from functional programming
(nested expressions, lazy evaluation, higher-order functions),
logic programming (logical variables, partial data structures,
built-in search), and concurrent programming (concurrent evaluation
of constraints with synchronization on logical variables).
The current \CYS implementation does not support concurrent
constraints. Alternatively, one can write distributed applications
by the use of sockets that can be registered and accessed
with symbolic names.
Moreover, \CYS also supports
the high-level implementation of
graphical user interfaces and web services
(as described in more detail in
\cite{Hanus99PPDP,Hanus00PADL,Hanus01PADL,Hanus06PPDP}).

We assume familiarity with the ideas and features
of Curry as described in the Curry language definition \cite{Hanus12Curry}.
Therefore, this document only explains the use of the different
components of \CYS
and the differences and restrictions of \CYS
(see Section~\ref{sec-restrictions})
compared with the language Curry (Version 0.8.3).
The basic ideas of the implementation of \CYS
can be found in
\cite{BrasselHanusPeemoellerReck11,BrasselHanusPeemoellerReck11WLP}.

\bigskip

\subsection*{Acknowledgements}

This work has been supported in part by the
DFG grants Ha 2457/5-1 and Ha 2457/5-2.
